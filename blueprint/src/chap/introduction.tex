\chapter{Introduction}
\label{section-introduction}

\noindent
This blueprint is a guide to the design and implementation of a series of simple tactics,
and some useful tips about Meta Programming.

\section{Lean 中元编程的基本概念}

Lean 的编译过程是先将一个字符串 (String) 解析 (Parse) 成一个 Lean 的 Syntax 对象,
也就是具体语法树 (Concrete Syntax Tree),
然后将语法树解释 (Elaborate) 为一个 Lean 的 Expr 对象,也就是抽象语法树 (Abstract Syntax Tree)。
注意在 Parse 阶段,Syntax 对象只包含了语法信息,而没有语义信息。 而在 Elaborate 阶段,
Lean 会根据上下文信息,将 Syntax Tree 中的每个节点解释为一个 Lean 一个带有具体含义的 Expr,
从而构建出一个能够被 Lean 的内核理解和操作的表达式对象。例如 Lean 的内核能够求出一个 Expr 的类型是哪个 Expr
,能够对 Expr 进行 lambda 演算的基本操作等等。

元编程 (Meta Programming) 主要是围绕 Expr 对象展开的。在证明的过程中,我们会构造很多的 Expr 对象,
希望用这些 Expr 作为材料来构造出一个定理的证明。Lean 提供了一系列的 API 来操作 Expr 对象,由于操作过程涉及到
对当前 ProofState 的查询以及修改,因此这些操作都是在 Tactic Monad 中进行的。

